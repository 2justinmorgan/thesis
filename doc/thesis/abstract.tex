The need for website administrators to efficiently and accurately detect the presence of web bots has shown to be a challenging problem. As the sophistication of modern web bots increases, specifically their ability to more closely mimic the behavior of humans, web bot detection schemes are more quickly becoming obsolete by failing to maintain effectiveness. Though machine learning-based detection schemes have been a successful approach to recent implementations, creators of web bots continuously adopt similar machine learning tactics to mimic human users, thus bypassing such detection schemes. This work seeks to address the issue of machine learning-based bots bypassing machine learning-based detection schemes, by introducing a novel unsupervised learning approach to cluster users based on behavioral biometrics. The idea is that, by differentiating users based on their behavior~\cite{FEHER201219}, for example how they use the mouse or type on the keyboard, information can be provided for website administrators to make more informed decisions on declaring if a user is a human or a bot. This approach is similar to how modern websites require users to login before browsing their website; which in doing so, website administrators can make informed decisions on declaring if a user is a human or a bot. An added benefit of this approach is that it is a human observational proof (HOP); meaning that it will not inconvenience the user (user friction) with human interactive proofs (HIP) such as CAPTCHA, or with login requirements.