
\chapter{Implementation}\label{ch:implementation}
This thesis work utilizes an approach that begins by identifying users, which include human and bot users, based solely on their mouse movement behavior~\cite{intrustion_detection_using_mouse_dynamics}.
Specifically, this project presents a user differentiation method based on mouse behavior metrics, such as movement angle, movement velocity, scroll velocity, etc., instead of relying on client IP addresses present in the server logs.
Further study could include metrics that are also used in previously implemented web bot detection schemes.
But this project will start with just the mouse behavior metrics.
The presented approach in this project implies that, upon identifying users based on their mouse use behavior, decisions to declare userX, which is a single cluster, as a human or a bot are reinforced by empirical evidence of web traffic patterns corresponding to userX.
This approach could be an improvement to the inaccuracies present in previous supervised learning-based web bot detection schemes.
Additionally, this "identify users first, then classify as human or bot" method is similar to the current industry standard of websites requiring all visitors to log into an account for further use of their website; which reinforces decisions to declare account-holderX as a human or a bot, regardless of the IP address of account-holderX.
However, this "log-in, then use website" method causes user friction~\cite{how_recaptcha_is_improving_user_experience}, a concept introduced and considered by the latest "covert" versions of reCAPTCHA, that implies the inconveniences a user must experience to prove they are human and not a bot.
An example of this could be requiring a user to click/check "I'm not a robot" on older versions of reCAPTCHA.
In conclusion, this project presents an unsupervised, clustering method to autonomously identify users, as if they were to log in to an account, providing a means to make more informed decisions of the "web bot-ness" of visitors on a website.

\input{chapters/implementation/dataset_section}


\section{Features Engineering}\label{sec:features-engineering}
The intrustion detection scheme~\cite{intrustion_detection_using_mouse_dynamics}, while also using the Balabit dataset as it was intended to be used, extracted a set of features from the raw mouse position data.
Though their work implemented a supervised learning-based binary classifier, the features they extracted were proven to be effective metrics in differentiating and identifying users.
Instead of using all 6 elements of a datapoint vector, as outlined in the \textit{Balabit Dataset} section, we elected to only use the client timestamp ($t$), x position ($x$), and y position ($y$) values.
These three values construct a triplet, ($t_i$, $x_i$, $y_i$), $i = 1{\dots}n$, where $n$ is the number of recorded mouse positions, or datapoint vectors, in a session file.
From these three values, or triplets, of a single datapoint vector, in the list of vectors of a session file, the following features were extracted:
\begin{itemize}
    \item \textbf{velocity}: $v_i = \frac{\Delta p_i}{\Delta t_i}$, where $\Delta p_i = \lvert p_{i+1} - p_i \rvert$ and $\Delta t_i = t_{i+1} - t_i$
    \item \textbf{horizontal velocity}: ${v_x}_i = \frac{\Delta x_i}{\Delta t_i}$, where $\Delta x_i = \lvert x_{i+1} - x_i \rvert$ and $\Delta t_i = t_{i+1} - t_i$
    \item \textbf{vertical velocity}: ${v_y}_i = \frac{\Delta y_i}{\Delta t_i}$, where $\Delta y_i = \lvert y_{i+1} - y_i \rvert$ and $\Delta t_i = t_{i+1} - t_i$
    \item \textbf{acceleration}: $a_i = \frac{\Delta v_i}{\Delta t_i}$, where $\Delta v_i = \lvert v_{i+1} - v_i \rvert$ and $\Delta t_i = t_{i+1} - t_i$
    \item \textbf{jerk}: $j_i = \frac{\Delta a_i}{\Delta t_i}$, where $\Delta a_i = \lvert a_{i+1} - a_i \rvert$ and $\Delta t_i = t_{i+1} - t_i$
    \item \textbf{theta}: $\Theta _i = \arctan 2(\frac{\Delta y_i}{\Delta x_i})$, where $\Delta y_i = \lvert y_{i+1} - y_i \rvert$ and $\Delta x_i = \lvert x_{i+1} - x_i \rvert$
\end{itemize}

\subsection{Realtime Generation}\label{subsec:realtime-generation}
As described in the objective, this implementation is meant to detect web bots and botnet attacks in realtime.
At this stage of the detection scheme, a program would need to be run in realtime to compute the 6 features outline above.
Initially, a Python program was used to generate these 6 features as described.
The program calculated all 1676 session files, from all 10 users, in an average of 7 minutes.
By pre-allocating lists of numeric values, and incrementing a counter variable that keeps track of where to insert the next calculated feature value into the list of numeric values, the runtime was reduced from 7 minutes to slightly more than 3 minutes.
Further, the entire features generator program was converted from Python to Golang.
By creating Go-routines on each of the 6 features, the average runtime of the Golang program calculating all 1676 sessions files was less than 30 seconds.
All runtimes for realtime feature generation do not include data cleaning and prep.
Since the Balabit dataset has many duplicate timestamp values, with different \textit{x} and \textit{y} values, a pre-generation step would need to take place to remove erooneous duplicates, as a means to "clean" the raw data input.

\subsection{Statistics}\label{subsec:statistics}
After the $n$ feature values have been generated, where $n$ is the number of datapoint vectors or triplets in a session, for each of the 6 features, the values would need to be represented with statistical values.
The statistical values used for each of the 6 features are \textbf{mean}, \textbf{median}, \textbf{mode}, \textbf{interquartile range}, \textbf{minimum}, \textbf{maximum}, \textbf{range}, and \textbf{standard deviation}.

It is worth noting that the \textit{mode} and \textit{minimum} values did not appear to be as useful as the other statistical metrics.
The minimum values of each of the feature values lists were mostly zero.
This is a result of the feature calculations.
For example, horizontal velocity could be zero if the $x$ position does not change in two successive datapoint vectors.
Formally, ${v_x}_i = \frac{\Delta x_i}{\Delta t_i} = 0$ if $\Delta x_i = \lvert x_i - x_{i+1} \rvert = 0$, meaning $x_i = x_{i+1}$.
This is one example of why the \textit{maximum} and \textit{range} values were often the same.
If $range = \vert maximum - minimum \rvert$, where $mimimum = 0$, than $range = maxmimum$.




\section{Clustering}\label{sec:clustering}
At this point in the web bot and botnet attack detection scheme, a number of metrics are obtained that can be used to differentiate users.
For each of the 6 features, 8 statistical metrics are calculated to represent their respective features, totaling to 48 values that can be used for clustering.
The purpose of clustering sessions, based on the outlined 48 values, is to differentiate users based solely on their mouse movement behavior.
Session files, or lists of mouse position vectors, would be inputted into this clustering algorithm to output clusters of differentiated users.
In a realtime environment, the session files would be the mouse position metrics that are periodically inputted from the user's machine, as described in the \textit{Realtime} subsection of the \textit{Balabit Dataset} section.

During the testing and analysis stages of this research, 2 clustering algorithms were used on the 1676 x 48 inputted session files.
With \textbf{k-means} clustering, the value for \textit{k} was set to 10, the number of users present in the 1676 session files. All 48 values pertaining to each session file, were used as the passed-in clustering features.
With \textbf{hierarchical} clustering, all 48 values were also used.




\section{Classification}\label{sec:classification}
Knowing that a certain user, differentiated by clustering, has a specific number of network requests, hard-coded request thresholds can be enforced.
Similar to threshold enforcement on user-based websites, the classification staged of this bot detection scheme will entail blacklisting IP addresses of users of a profile, or cluster, users exceed network request thresholds.
However, users would not need to login, as they would on a user-based website, since all users are autonomously differentiated in the clustering stage of this thesis implementation.

Differentiating users by their mouse movement behavior will enable additional classification schemes.
The evaluation chapter~\ref{ch:evaluation} outlines how accurate the clustering implementation in~\ref{sec:clustering} differentiates users so this classification stage will occur.
You will see that, at this current state in the thesis research, users profiles are not distinctly clustered for a classification step to occur.

