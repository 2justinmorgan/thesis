
\section{Features Engineering}\label{sec:features-engineering}
The intrustion detection scheme~\cite{intrustion_detection_using_mouse_dynamics}, while also using the Balabit dataset as it was intended to be used, extracted a set of features from the raw mouse position data.
Though their work implemented a supervised learning-based binary classifier, the features they extracted were proven to be effective metrics in differentiating and identifying users.
Instead of using all 6 elements of a datapoint vector, as outlined in the \textit{Balabit Dataset} section~\ref{subsec:balabit-dataset}, we elected to only use the client timestamp ($t$), x position ($x$), and y position ($y$) values.
These three values construct a triplet, ($t_i$, $x_i$, $y_i$), $i = 1{\dots}n$, where $n$ is the number of recorded mouse positions, or datapoint vectors, in a session file.
The three values, time and 2d coordinates, of a mouse position datapoint are all that is needed to generate the mouse movement features in this thesis research.
From these three values, or triplets, of a single datapoint vector, in the list of vectors of a session file, the following features were extracted:
\begin{itemize}
    \item \textbf{velocity}: $v_i = \frac{\Delta p_i}{\Delta t_i}$, where $\Delta p_i = \lvert p_{i+1} - p_i \rvert$ and $\Delta t_i = t_{i+1} - t_i$
    \item \textbf{horizontal velocity}: ${v_x}_i = \frac{\Delta x_i}{\Delta t_i}$, where $\Delta x_i = \lvert x_{i+1} - x_i \rvert$ and $\Delta t_i = t_{i+1} - t_i$
    \item \textbf{vertical velocity}: ${v_y}_i = \frac{\Delta y_i}{\Delta t_i}$, where $\Delta y_i = \lvert y_{i+1} - y_i \rvert$ and $\Delta t_i = t_{i+1} - t_i$
    \item \textbf{acceleration}: $a_i = \frac{\Delta v_i}{\Delta t_i}$, where $\Delta v_i = \lvert v_{i+1} - v_i \rvert$ and $\Delta t_i = t_{i+1} - t_i$
    \item \textbf{jerk}: $j_i = \frac{\Delta a_i}{\Delta t_i}$, where $\Delta a_i = \lvert a_{i+1} - a_i \rvert$ and $\Delta t_i = t_{i+1} - t_i$
    \item \textbf{theta}: $\Theta _i = \arctan 2(\frac{\Delta y_i}{\Delta x_i})$, where $\Delta y_i = \lvert y_{i+1} - y_i \rvert$ and $\Delta x_i = \lvert x_{i+1} - x_i \rvert$
\end{itemize}

\subsection{Realtime Generation}\label{subsec:realtime-generation}
As described in the objective, this implementation is meant to detect web bots and botnet attacks in realtime.
At this stage of the detection scheme, a program would need to be run in realtime to compute the 6 features outline above.
Initially, a Python program was used to generate these 6 features as described.
The program calculated all 1676 session files, from all 10 users, in an average of 7 minutes.
By pre-allocating lists of numeric values, and incrementing a counter variable that keeps track of where to insert the next calculated feature value into the list of numeric values, the runtime was reduced from 7 minutes to slightly more than 3 minutes.
Further, the entire features generator program was converted from Python to Golang.
By creating Go-routines on each of the 6 features, the average runtime of the Golang program calculating all 1676 sessions files was less than 30 seconds.
All runtimes for realtime feature generation do not include data cleaning and prep.
Since the Balabit dataset has many duplicate timestamp values, with different \textit{x} and \textit{y} values, a pre-generation step would need to take place to remove erooneous duplicates, as a means to "clean" the raw data input.

\subsection{Statistics}\label{subsec:statistics}
After the $n$ feature values have been generated, where $n$ is the number of datapoint vectors or triplets in a session, for each of the 6 features, the values would need to be represented with statistical values.
The statistical values used for each of the 6 features are \textbf{mean}, \textbf{median}, \textbf{mode}, \textbf{interquartile range}, \textbf{minimum}, \textbf{maximum}, \textbf{range}, and \textbf{standard deviation}.

It is worth noting that the \textit{mode} and \textit{minimum} values did not appear to be as useful as the other statistical metrics.
The minimum values of each of the feature values lists were mostly zero.
This is a result of the feature calculations.
For example, horizontal velocity could be zero if the $x$ position does not change in two successive datapoint vectors.
Formally, ${v_x}_i = \frac{\Delta x_i}{\Delta t_i} = 0$ if $\Delta x_i = \lvert x_i - x_{i+1} \rvert = 0$, meaning $x_i = x_{i+1}$.
This is one example of why the \textit{maximum} and \textit{range} values were often the same.
If $range = \vert maximum - minimum \rvert$, where $mimimum = 0$, than $range = maxmimum$.

