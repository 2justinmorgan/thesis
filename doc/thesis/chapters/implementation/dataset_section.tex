
\section{Dataset}\label{sec:dataset}
In order to cluster users based on their mouse movement behavior, data of their mouse movement must be obtained.
An ideal scenario for this research would be to record the mouse movement of a high number of users, while navigating a specified user interface, thus creating a mouse movement recordings dataset.
This would be ideal because there would be more user profiles, as well as knowledge of the graphical user interface used to generate these user profiles.
Unfortunately, doing so would require time, people, and resources that are not available during time-frame of this thesis research.
However, there are alternatives that maintain the validity of this research, such as using a predefined dataset with mouse behavior-based user profiles.

\subsection{Balabit Dataset}\label{subsec:balabit-dataset}
For testing and analysis purposes, a predefined dataset of users were used in this research.
The Balabit Mouse Challenge Dataset~\cite{balabit_dataset} was the primary dataset used in this research.
This dataset includes timing and positioning information of a web user's mouse pointer.
The authors of the dataset advertise that it can be used for authentication and identification purposes.
Researchers with focus on creating and evaluating the performance of behavioral biometric algorithms, which in this case draws from the mouse movement metrics of a user, are an intended audience for this publicly accessible dataset.
Originating from a data science competition on datapallet.io, the dataset is helpful to researchers and experts in the fields of IT security and datascience.

The competition for which the Balabit dataset originates from included a challenge of protecting users from unauthorized accesses into their accounts.
When users would login to their account, located on a remote server, recording their mouse movement behavior was a necessary step in an effort to increase account security.
Supposing that the method for which a user moves their mouse was unique to that user, a sort of biometric identifier can be obtained for account user authenticity.
If the mouse movement characteristics of a user, in a particular session, does not match the recorded and expected characteristics of the account holder, than that user in the particular session is said to be an unauthorized accessor.
In order to apply such a intrusion detection schema, a supervised learning-based model would need to be built and utilized.
However, this research does not intend on detecting unauthorized accessors, nor does it intend on using supervised learning in its implementation.

Although the Balabit dataset was intended to be used for creating and evaluating supervised learning-based models, the dataset still contains valuable user profiles and session recordings.
These user profiles are defined by a series of session recordings that are labeled with a single user of an account.
There are about 100 to 200 sessions recordings, spread over 10 users, with an average of 15 minutes of recording time per session.
Session recordings are split into two sets: training and testing.
This is for supervised learning uses.
For this research, all session files were combined into one set that is to be clustered and analyzed.
A session record is a csv file containing these fields~\cite{balabit_dataset}:
\begin{itemize}
    \item \textbf{record timestamp}: elapsed time (in seconds) since the start of the session as recorded by the network monitoring device used in the creation of the Balabit dataset
    \item \textbf{client timestamp}: elapsed time (in seconds) since the start of the session as recorded by the RDP client used by each of the 10 users
    \item \textbf{button}: the current condition of the mouse buttons
    \item \textbf{state}: additional information about the current state of the mouse
    \item \textbf{x}: the x coordinate (in pixels) of the mouse cursor on the screen
    \item \textbf{y}: the y coordinate (in pixels) of the mouse cursor on the screen
\end{itemize}
An example of a session record looks like:
\begin{table}
    \centering
    \caption{Balabit dataset session file format}
    \bigskip
    \label{tab:balabit-dataset-format}
    \begin{tabular}{ |c|c|c|c|c|c| }
        \hline
        \textbf{record timestamp} & \textbf{client timestamp} & \textbf{button} & \textbf{state} & \textbf{x} & \textbf{y} \\
        \hline
        0.0 & 0.0 & NoButton & Move & 399 & 962 \\
        0.157999992371 & 0.155999999959 & NoButton & Move & 402 & 962 \\
        0.365999937057 & 0.248999999953 & NoButton & Move & 407 & 962 \\
        0.365999937057 & 0.358000000007 & Left & Pressed & 430 & 962 \\
        0.476999998093 & 0.467999999993 & Left & Released & 474 & 963 \\
        \ldots & \ldots & \ldots & \ldots & \ldots & \ldots \\
        \hline
    \end{tabular}
\end{table}
%duplicate session_0335985747

\subsection{Realtime Dataset}\label{subsec:realtime-dataset}
In a realtime environment, where the user-differentiating algorithm is deployed, mouse movement data would come from the browser on the user's computer.
By using a JavaScript mousemove event listener, the coordinates of a user's mouse can be determined and recorded while a user is in session.
On average, a computers mouse position is polled 125 times per second~\cite{mouse_dpi_and_polling_rate_explained}~\cite{mouse_dpi_and_usb_polling_rate}.
This means that if a 10 second user session is recorded, there should be an average of 1,250 mouse position records of any single user browsing a website.
Records can be in a csv format:
\begin{itemize}
    \item \textbf{time}: elapsed time (in seconds) since the start of the session as recorded by the user's browser
    \item \textbf{x}: the x coordinate (in pixels) of the mouse cursor on the screen
    \item \textbf{y}: the y coordinate (in pixels) of the mouse cursor on the screen
\end{itemize}
A typical recorded session of a user's mouse movement metrics would look like:
\begin{table}
    \centering
    \caption{Realtime session file format}
    \bigskip
    \label{tab:realtime-dataset-format}
    \begin{tabular}{ |c|c|c| }
        \hline
        \textbf{time} & \textbf{x} & \textbf{y} \\
        \hline
        0.0 & 241 & 93 \\
        0.008121 & 278 & 77 \\
        0.015828 & 291 & 54 \\
        0.026942 & 302 & 48 \\
        0.037201 & 317 & 50 \\
        \ldots & \ldots & \ldots \\
        \hline
    \end{tabular}
 \end{table}
These records can be stored on the user's computer, most likely in the browser via a JavaScript variable, then periodically sent to the server for which the website is hosted.
From this input, on the server, the web bot and botnet detection scheme will begin.
Pseudo code that generalizes the process of obtaining and sending a user's mouse movement metrics looks like:

{\color{darkgray} // the list of "times" "x" and "y" values POSTed to the server}
\newline
{\color{blue} \textbf{var}} records;

{\color{blue} \textbf{function}} flushRecords(bufSize)
\newline
\hspace*{25pt} {\color{darkgray} // reallocate "bufSize" number of indices for "times" "x" and "y" lists}

{\color{blue} \textbf{function}} postAndFlushRecords(bufSize)
\newline
\hspace*{25pt} {\color{darkgray} // POST all recorded "times" "x" and "y" values to the server}

{\color{blue} \textbf{function}} log(elapsedTime, x, y, bufSize)
\newline
\hspace*{25pt} {\color{darkgray} // insert "elapsedTime" "x" and "y" values into "records"}
\newline
\hspace*{25pt} {\color{darkgray} // postAndFlushRecords() if there are "bufSize" number of records}

{\color{blue} \textbf{function}} initBufferTimeout(timeLimit, bufSize)
\newline
\hspace*{25pt} {\color{darkgray} // postAndFlushRecords() every "timeLimit" duration}

{\color{blue} \textbf{function}} init()
\newline
\hspace*{25pt} {\color{darkgray} // use the "window" object's "onmousemove" func to log() mouse positions}
\newline
\hspace*{25pt} {\color{darkgray} // initBufferTimeout() to periodically POST logged "records" to the server}

{\color{darkgray} // called once upon every page load}
\newline
init();

The actual code, client{\_}mouse{\_}tracker.js, can be found in the src/ dir on the remote repo~\cite{thesis_github_repo} of this research.

