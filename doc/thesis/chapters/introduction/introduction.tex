
\chapter{Introduction}
A web bot is a programmatic simulation of a human user web browsing session. Similarly, a botnet is a collection of web bots working in unison. Most of these bots are programmed with the intention of foregoing remedial and repetitive tasks that a human user would be required to do. Some examples of these tasks may include searching for specific items, with a given criteria, through a large number of items on an ecommerce site such as Amazon. Another example can be the action of downloading lists textual data on a web page that would require the user to do so manually. There are many use cases for these programmatic simulations, otherwise known as web bots. Unfortunately, web bots can be used maliciously or irresponsibly, thus introducing a number of problems for the users and administrators of a web site~\cite{1ee426975c3d46d2ba6ef5c2d76384c5}~\cite{bad_bot_report}. Efforts to detect these web bots have proven to be successful~\cite{akamai_bot_detection}~\cite{Hamidzadeh2018}~\cite{ZABIHIMAYVAN2017129}. However, due to the increasing sophistication of web bots, said detection schemes are often bypassed and deemed obsolete~\cite{ROVETTA2020102577}~\cite{STEVANOVIC2013698}~\cite{10.1109/DSN.2013.6575366}. This work addresses some of these issues.
